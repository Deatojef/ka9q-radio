\chapter{README}
\hypertarget{md__2Users_2karn_2src_2ka9q-radio_2README}{}\label{md__2Users_2karn_2src_2ka9q-radio_2README}\index{README@{README}}
{\itshape ka9q-\/radio} is a software defined radio for Linux I\textquotesingle{}ve been working on for a few years. It is very different from most other amateur SDRs in several respects\+:


\begin{DoxyEnumerate}
\item Efficient multichannel reception. A single Raspberry Pi 4 can simultaneously demodulate, in real time, every NBFM channel on a VHF/\+UHF band (i.\+e., several hundred) with plenty of real time left over.
\end{DoxyEnumerate}

A mid-\/range x86 can handle the RX888 Mk\+II at full sample rate (129.\+6 MHz), receiving multiple channels sumultaneously on all LF/\+MF/\+HF ham bands (or anything else) plus 6 meters.

A Raspberry Pi 5 can handle the RX888 Mk\+II at half sample rate (64.\+8 MHz), enough to receive up to 30 MHz. (An external hardware anti-\/alias filter is required, and "{}\+FFTW Wisdom"{} must be generated for the Pi5 to run this fast.)


\begin{DoxyEnumerate}
\item All I/O (both signal and control/status) uses IP multicasting. This makes it easy for more than one module, on the same computer or on a LAN, to operate on the outputs of other modules, or for individual modules to be restarted without restarting everything else.
\end{DoxyEnumerate}

If you want a user-\/friendly, interactive, graphics-\/laden SDR with a simple learning curve, then {\itshape ka9q-\/radio} is NOT what you\textquotesingle{}re looking for! (At least not yet.) Try one of the many excellent SDR programs already available like SDR\#, Cubic SDR, gqrx, etc, or the standalone Kiwi SDR. This is my personal experiment in building a very different kind of SDR that runs as a component serving other applications.

The core components in {\itshape ka9q-\/radio} run as Linux \textquotesingle{}daemons\textquotesingle{} (background programs) with little (or no) user interaction. Turnkey systems can be configured to, e.\+g., demodulate and record every FM channel on a band, or decode, log and/or relay digital messages (e.\+g., APRS, FT-\/8, WSPR, Horus 4FSK, radiosondes). These programs are automatically launched by the (new) Linux standard system manager program {\itshape systemd}.

The core component is the {\itshape radiod} daemon. It reads an A/D stream directly from a front end and executes a configured set of digital downconverters and simple demodulators for various linear and FM modes, including AM, SSB, CW and a raw IQ mode intended mainly for use by other programs.

Previous versions (before mid-\/2023) of ka9q-\/radio had separate programs (e.\+g., {\itshape airspyd}) for talking to several makes of SDR front end hardware and generated an I/Q multicast stream for {\itshape radiod}. Because of performance problems, code and configuration complexity and general lack of utility these separate programs have been obsoleted and the front end drivers built directly into {\itshape radiod}. Support is currently provided for generic RTL-\/\+SDR dongles, the Airspy R2, Airspy HF+, AMSAT UK Funcube Pro+ and RX-\/888 Mk II. A synthetic front end, {\itshape sig\+\_\+gen}, is also provided. It simulates a front end, either complex or real, producing gaussian noise and single carrier at specified amplitudes. It can also transmit my WWV/H simulator {\itshape wwvsim}, but it\textquotesingle{}s not yet well integrated, mainly because of the need for an external speech synthesizer.

Support will be forthcoming for the SDRPlay and the Hack\+RF (receive only).

Two very rudimentary programs are provided for interactive use; {\itshape monitor} listens to one or more demodulated audio streams and {\itshape control} controls and displays the status of a selected receiver channel. It can also create and delete dynamic channel instances. The {\itshape control} program uses a flexible and extensible metadata protocol that could be (and I hope will be) implemented by much more sophisticated user interfaces. Various utilities are provided to record or play back signal streams, compress PCM audio into Opus, pipe a stream into digital demodulators, etc.

{\itshape Radiod} now periodically multicasts ("{}beacons"{}) status information on each output stream and user programs are being enhanced to make use of it. For example, {\itshape monitor} now displays the frequency, mode and signal-\/to-\/noise ratio of each channe.

Although I\textquotesingle{}ve been running all this myself for several years, it is NOT yet ready for general use. A LOT of work still remains, especially documentation. But you\textquotesingle{}re welcome to look at it, make comments and even try it out if you\textquotesingle{}re feeling brave. I would especially like to hear from those interested in building it into their own SDR applications.

My big inspiration for the multichannel part of my project was this most excellent paper by Mark Borgerding\+: "{}\+Turning Overlap-\/\+Save into a \+Multiband Mixing, Downsampling Filter Bank"{}. You probably won\textquotesingle{}t understand how it works until you\textquotesingle{}ve read it\+:

\href{https://www.iro.umontreal.ca/~mignotte/IFT3205/Documents/TipsAndTricks/MultibandFilterbank.pdf}{\texttt{ https\+://www.\+iro.\+umontreal.\+ca/\texorpdfstring{$\sim$}{\string~}mignotte/\+IFT3205/\+Documents/\+Tips\+And\+Tricks/\+Multiband\+Filterbank.\+pdf}}

Although there are other ways to build efficient multichannel receivers, most notably the polyphase filter bank, fast convolution is extraordinarily flexible. Each channel is independently tunable with its own sample rate and filter response curve. The only requirement is that the impulse response of the channel filters be shorter than the (configurable) overlap interval in the forward FFT.

Updated 11 April 2024 ~\newline
 Phil Karn, KA9Q ~\newline
 \href{mailto:karn@ka9q.net}{\texttt{ karn@ka9q.\+net}} 